\documentclass[11pt,oneside,a4paper,notitlepage]{article}

\usepackage[utf8]{inputenc}
\usepackage[ngerman]{babel}
\usepackage[margin=1.5cm]{geometry}

%kommentare, zitate, quellcode
\usepackage{verbatim}
%\fontfamily{sfdefault}
\renewcommand{\familydefault}{\sfdefault}
%
\usepackage{graphicx}
%fuer tabellen
\usepackage{tabularx}
\usepackage{tabulary}
%
%formatierung listen
\let\oldenumerate\enumerate
\renewcommand{\enumerate}{
  \oldenumerate
  \setlength{\itemsep}{1pt}
  \setlength{\parskip}{0pt}
  \setlength{\parsep}{0pt}
}

%
%referenzen und links
\usepackage{hyperref}
\hypersetup{
colorlinks=true,
linkcolor=cyan,
urlcolor=cyan,
hidelinks=false
}
%
% 
\renewcommand{\arraystretch}{1.5}
%

\begin{document}
%
\section{EMail als Clientstruktur }
%
Die Nachrichten an die Fachangestellten könnten über bestehende Infrastruktur zum Email Versand zugestellt werden. Dazu würde der Buchhaltungs/Verwaltungsclient bei seiner Anfrage 
an einen Fachangestellten einen Prozess auslösen bei dem eine Html formatierte EMail mit einem Eingabeformular an einen Fachangestellten versendet wird. Dieser kann in seinem 
EMail Client die Nachricht öffnen, die zugehörigen Werte in das Formular eingeben und die Antwort an den Dienst zurücksenden. Der Dienst extrahiert aus der EMail Datei den
entsprechenden Vorgang, die Attribute und Werte und konvertiert diese in das im System genutzte Format. 

%
\subsection*{Vorteile}

Durch die Nutzung von EMail zur Übermittlung kann bestehende Kommunikationsinfrastruktur benutzt werden für welche die Empfänger zudem bereits zum Empfang und Versand nötige Anwendung installiert haben.

%
\subsection*{Nachteile}

\begin{itemize}
\item ggf aufwändig zu parsen
Die Nutzung erzeugt Abhängigkeiten zur eingesetzten EMail Infrastruktur
Desweiteren werden Html formatierte EMails und insbesondere Formulare werden von im Geschäftsumfeld genutzten Email-Clients nicht hinreichend unterstützt. \href{}{Quelle1} \href{}{Quelle2}
EIn DIenst der 
Dies würde aufwändige Anpassungen notwendig machen.

%
\end{document}

