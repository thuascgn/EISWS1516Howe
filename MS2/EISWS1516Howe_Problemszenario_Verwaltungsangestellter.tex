\documentclass[11pt,oneside,a4paper,notitlepage]{article}

\usepackage[utf8]{inputenc}
\usepackage[ngerman]{babel}
\usepackage[margin=1.5cm]{geometry}

%kommentare, zitate, quellcode
\usepackage{verbatim}
%\fontfamily{sfdefault}
\renewcommand{\familydefault}{\sfdefault}
%
\usepackage{graphicx}
%fuer tabellen
\usepackage{tabularx}
\usepackage{tabulary}
%
%formatierung listen
\let\oldenumerate\enumerate
\renewcommand{\enumerate}{
  \oldenumerate
  \setlength{\itemsep}{1pt}
  \setlength{\parskip}{0pt}
  \setlength{\parsep}{0pt}
}

%
%referenzen und links
\usepackage{hyperref}
\hypersetup{
colorlinks=true,
linkcolor=cyan,
urlcolor=cyan,
hidelinks=false
}
%
% 
\renewcommand{\arraystretch}{1.5}
%

\begin{document}
%
\section{Problemszenario - Verwaltungsangestellter}


\textbf{Titel: }\\
Erzeugen eines Geschäftsvorgangs\\
\textbf{Personen: }\\
Hr. Fillipi, Buchhalter in einem mittelgroßen Unternehmen\\
Fr. Kiesewetter, Abteilungskollegin von Herrn Fillipi\\
Hr. Schell, Kollege aus der Marketingabteilung
\\[1cm]
\noindent
\textbf{Narrativ: }\\
Nach einem kurzfristigen Telefonkonferenz mit zwei seiner Abteilungskollegen und seiner Vorgesetztem, die in dieser Woche 
in einer Zweigstelle des Unternehmens weilt, setzt sich Herr Fillipi wieder an seinen Schreibtisch und öffnet sein Fenster des Buchhaltungssystems. Er wartet einen zurzeit laufenden Aktualisierungsvorgang seines Dokumenteneingangs ab und sieht eine Reihe von neuen Dokumenten die die Aushilfen aus der Erfassungsabteilung bereitgestellt haben. \\
%Zu den Aufgaben von Hr Fillipi gehört 
Bei den Absendern der Dokumente erkennt er den Firmennamen eines dienstleistenden Messebauers, von dem in der Telefonkonferenz die Rede war, und wählt eines der Dokumente aus um dazu einen buchhalterischen Vorgang zu beginnen. In einer Detailansicht des Dokuments überfliegt er kurz das Dokument. Da der Messebauer nahezu ausschliesslich im Rahmen Marketingaktivitäten beauftragt wird weist Herr Fillipi dem Vorgang zunächst die Marketingabteilung zu und dann das zugehörige Projekt. Aber anscheinend ist kein passendes Projekt im System eingepflegt. Er greift zum Hörer und ruft den Kollegen Schell, von dem er weiß das dieser auch mit Messen beschäftigt ist, an um nach dem Projekt zu fragen. Leider nimmt dieser nicht ab. Er legt den Hörer auf und fragt Fr. Kiesewetter am Nebentisch ob sie Herrn Schell heute schon gesehen hat. Fr. Kiesewetter teilt ihm mit das sie gehört hat das Schell bis zum Nachmittag in einer Abteilungsbesprechung weilt. Hr Fillipi schliesst daraufhin den zuvor geöffneten Buchhaltungsvorgang.\\
Für eine Abteilungsbesprechung am Anfang nächster Woche soll er noch einige Rahmendaten der Buchhaltung sammeln. Er geht die vergangenen Vorgänge durch und überträgt Daten zu Eingangsdatum von Dokumenten, Anzahl von Rechnungen von wichtiger Dienstleister und Zulieferer in ein für diese Besprechnung vorbereitetes Excelformular.\\
Nach einer Weile klingelt sein Telefon und Hr Schell meldet sich das zur Zeit eine Besprechungspause machen und er Hrn Fillipis Anruf in Abwesenheit gesehen hat. Während Hr Fillipi nach dem Projekt fragt wechselt er gleichzeitig wieder in den geschlossenen Vorgang. Hr Schell teilt ihm derweil irritiert mit das die Abrechnungen für die Messe erst Anfang des nächsten Quartals kommen sollen aber das der Messebauer wohl für eine Roadshow beauftragt wurde welche wiederum vom Vertrieb organisiert wurde. Sie beenden das Telefonat und Hrn Fillipi weist dem Vorgang zunächst die Vertriebsabteilung zu und kann daraufhin aus einer Liste von Vertriebsprojekten das zugehörige Roadshow Projekt zuweisen und den Vorgang an die Kollegen der Vertriebsabteilung weitergeben.







%
\end{document}

