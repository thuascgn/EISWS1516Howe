\documentclass[11pt,oneside,a4paper,notitlepage]{article}

\usepackage[utf8]{inputenc}
\usepackage[ngerman]{babel}
\usepackage[margin=1.5cm]{geometry}

%kommentare, zitate, quellcode
\usepackage{verbatim}
%\fontfamily{sfdefault}
\renewcommand{\familydefault}{\sfdefault}
%
\usepackage{graphicx}
%fuer tabellen
\usepackage{tabularx}
\usepackage{tabulary}
%
%formatierung listen
\let\oldenumerate\enumerate
\renewcommand{\enumerate}{
  \oldenumerate
  \setlength{\itemsep}{1pt}
  \setlength{\parskip}{0pt}
  \setlength{\parsep}{0pt}
}

%
%referenzen und links
\usepackage{hyperref}
\hypersetup{
colorlinks=true,
linkcolor=cyan,
urlcolor=cyan,
hidelinks=false
}
%
% 
\renewcommand{\arraystretch}{1.5}
%

\begin{document}
%
\section{Proof of Concepts - Spezifikation}
%
\subsection{Nachrichtenpriorisierung}
Es soll eine Struktur zum temporären Speichern und Ausliefern von Nachrichten gefunden werden welche die folgende Anforderungen erfüllt:

\begin{enumerate}
\item Sie ist auf einem Webserver lauffähig
\item auf dem passenden Webserver können zügig grundlegende API Funktionen implementiert werden
\item Speichern von xml/json als Key-Value Par
\item Eine Abrufpriorisierung der gespeicherten Tupel muss so möglich sein das die Anfrage eines Client mit dem Aufruf /ressource?=next die korrekten Werte erhalten
\item es soll vermieden werden die Priorisierungsparameter serverseitig in den Aufruf zu injezieren
\begin{enumerate}
\item Sortierung der Werte innerhalb der Speicherstruktur
\item Sortierung durch Veränderung der gespeicherten Werte
\end{enumerate}
\item 
\item 
\end{enumerate}
\noindent
%

%
\subsection{Regel Engine}
Es soll ein prototypisches Programm entwickelt werden das die grundlegenden Funktionalitäten der Regel-Engine abbildet und folgende Anforderungen erfüllt:

\begin{enumerate}
\item es muss aus einem Eingabeordner xml und/oder json Dateien eingelesen werden
\item die Informationen dieser Dateien werden auf eine Datenstruktur abgebildet
\item abhängig vom Informationen wird die Datenstruktur verändert
\item die veränderte Datenstruktur wird in einen Ausgabeordner geschrieben
\end{enumerate}
\noindent
%

\subsection{Desktop Clients}
%
Es soll ein protoypisches Programm entwickelt mit dem grundlegende Nutzungsanforderungen getestet werden und das folgende Anforderungen erfüllt:

\begin{enumerate}
\item es ist auf der Zielplatform der Clients PC mit Windows Betriebssystem lauffähig
\item es wird möglichst nativ für die Zielplatform entwickelt
\item es folgt dem WIMP Paradigma
\item es sind folgende Interaktionsstile(?)/pattern mit sichtbare Auswirkung der Interaktion möglich
\begin{enumerate}
\item Auswahl aus Listen 
\item Eingabe von Text
\item Klicken von Schaltflächen
\end{enumerate}
\item Bonus:
\begin{enumerate}
\item das Programm interagiert mit einem Webserver mit HTTP Methoden GET und POST Methoden
\end{enumerate}
\end{enumerate}
\noindent



\end{document}

