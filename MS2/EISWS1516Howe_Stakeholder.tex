\documentclass[11pt,oneside,a4paper,notitlepage]{article}

\usepackage[utf8]{inputenc}
\usepackage[ngerman]{babel}
\usepackage[margin=1.5cm]{geometry}

%kommentare, zitate, quellcode
\usepackage{verbatim}
%\fontfamily{sfdefault}
\renewcommand{\familydefault}{\sfdefault}
%
\usepackage{graphicx}
%fuer tabellen
\usepackage{tabularx}
\usepackage{tabulary}
%
%formatierung listen
\let\oldenumerate\enumerate
\renewcommand{\enumerate}{
  \oldenumerate
  \setlength{\itemsep}{1pt}
  \setlength{\parskip}{0pt}
  \setlength{\parsep}{0pt}
}

%
%referenzen und links
\usepackage{hyperref}
\hypersetup{
colorlinks=false,
hidelinks=false
}
%
% 
\renewcommand{\arraystretch}{1.5}
%

\begin{document}

\section{Stakeholder}

Aus der Domaenen- und Marktrecherche lassen sich die folgenden Stakeholder identifizieren: Scanner, Erfasser, Verwaltungsangestellte, Fachangestellte und Prozessverantwortliche bzw. Entscheider für die Verwaltungsabteilung oder eine Fachabteilung.\\

%
\paragraph*{Digitalisierer, ... }
\begin{center}
\begin{tabular}{| l | l | l | p{5cm} |}
%
\hline
Beziehung & Merkmal & Erwartung & Erfordnernis\\
\hline 
Interesse & Informationsausgabe & direkte Rückmeldung über Qualität des Dokumentenscan &  Erf...\\
%
\hline
\end{tabular}
\end{center}
%

% Erfasser, Erfassungshilfe, Erfassungskraft
\paragraph*{Korrekteure}
\begin{center}
\begin{tabular}{| l | l | l | p{5cm} |}
\hline
Beziehung & Merkmal & Erwartung & Erfordnernis\\
\hline
% Erfasser 
Interesse & Informationsausgabe & parallele Vergleichsmöglichkeiten von Scan, OCR Ergebnis & ... \\
%
Anspruch & Eingabe & zügige Eingabemöglichkeit von Korrekturen  & ... \\
\hline
\end{tabular}
\end{center}
%
%
\paragraph*{Verwaltungsangestellte}
\begin{center}
\begin{tabular}{| l | l | l | p{5cm} |}
\hline
Beziehung & Merkmal & Erwartung & Erfordnernis\\
\hline
Anspruch & Ausgabe & Information für einzelne Verwaltungs- und Verarbeitungsprozesse fokussiert & ...\\
Anspruch & Ausgabe & Überblick über den Gesamtzustand & ..\\
Anrecht & Dokumente & Reduktion der zu verarbeitenden Menge bei gleichbleibender Produktivität\\
Interesse & Verarbeitungsschritte & Reduktion & ... \\
Interesse & Attributierung & Reduzierung in Quantität und Komplexität der Vorgänge & ... \\   
Anrecht & Kommunikation & Reduktion des Aufwands für Abstimmung mit Kollegen & ... \\
\hline
\end{tabular}
\end{center}
%
\paragraph*{Fachangestellte}
\begin{center}
\begin{tabular}{| l | l | l | p{5cm} |}
\hline
Beziehung & Merkmal & Erwartung & Erfordnernis\\
\hline
Anspruch & Ausgabe & Informationsmenge auf Ihren Fachbereich fokussiert & ... \\
Interesse & Eingabe & Reduzierung zurückgreifen auf sekundäre Informationsquellen & ...\\
Anrecht & Dokumente & Reduktion der zu verarbeitenden Menge bei gleichbleibender Produktivität & ... \\
Interesse & Verarbeitungsschritte & Reduktion\\
Interesse & Attributierung & Reduzierung in Quantität und Komplexität der Vorgänge \\   
Anrecht & Kommunikation & Reduktion des Aufwands für Abstimmung mit Kollegen\\
%
\hline
\end{tabular}
\end{center}
%
%
\paragraph*{Abteilungsverantwortlicher Verwaltung ... }
\begin{center}
\begin{tabular}{| l | l | l | p{5cm} |}
\hline
Beziehung & Merkmal & Erwartung & Erfordnernis\\
\hline
Anspruch & Prozesszustand & Entscheidungsrelevante Informationen zu erhalten & ...\\
Anrecht & Informationen zum Systemzustand &  & ... \\
Anspruch & Prozesssteuerung & flexibel, zeitnah und wirkungsvoll in Abläufe eingreifen & ...\\
Interesse & \parbox[t]{4cm}{Schnittstellen zw\\Angestellten und System} & produktive Nutzung & ...\\
Interesse & \parbox[t]{4cm}{Schnittstellen zw\\Angestellten} & ermöglichen effizienter, fehlerarmer Informationsfluss & ...\\
%
\hline
\end{tabular}
\end{center}
%

%
\paragraph*{Geschäftsführung}
\begin{center}
\begin{tabular}{| p{3cm} | l | l | p{5cm} |}
\hline
Beziehung & Merkmal & Erwartung & Erfordnernis\\
\hline
%
Anspruch &  &  & ... \\
%
\hline
\end{tabular}
\end{center}
%

%
\paragraph*{Verwaltungsangestellte externer Unternehmen}
%
\begin{center}
\begin{tabular}{| l | l | l | p{5cm} |}
\hline
Beziehung & Merkmal & Erwartung & Erfordnernis\\
\hline
 & & &  \\
%
\hline
\end{tabular}
\end{center}




\end{document}

