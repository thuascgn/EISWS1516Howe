\documentclass[11pt,oneside,a4paper,notitlepage]{article}

\usepackage[utf8]{inputenc}
\usepackage[ngerman]{babel}
\usepackage[margin=1.5cm]{geometry}

%kommentare, zitate, quellcode
\usepackage{verbatim}
%\fontfamily{sfdefault}
\renewcommand{\familydefault}{\sfdefault}
%
\usepackage{graphicx}
%fuer tabellen
\usepackage{tabularx}
\usepackage{tabulary}
%
%formatierung listen
\let\oldenumerate\enumerate
\renewcommand{\enumerate}{
  \oldenumerate
  \setlength{\itemsep}{1pt}
  \setlength{\parskip}{0pt}
  \setlength{\parsep}{0pt}
}

%
%referenzen und links
\usepackage{hyperref}
\hypersetup{
colorlinks=false,
hidelinks=false
}
%
% 
\renewcommand{\arraystretch}{1.5}
%

\begin{document}

\section{Stakeholder}

Aus der Domaenen- und Marktrecherche lassen sich die folgenden Stakeholder identifizieren: Scanner, Erfasser, Verwaltungsangestellte, Fachangestellte und Prozessverantwortliche bzw. Entscheider für die Verwaltungsabteilung oder eine Fachabteilung.\\

\begin{center}
\begin{tabular}{| l | l | l | p{5cm} |}
\hline
% Erfordernis, 
Bezeichnung & Beziehung & Merkmal & Erwartung \\
\hline 
Scanner & & & \\
\hline 
Erfasser & & & \\
\hline
Verwaltungsangestellte & Interesse & Ausgabe &
 Information für einzelne Verwaltungs- und Verarbeitungsprozesse fokussiert\\
%
 & Anpruch & Ausgabe & 
 Überblick über den Gesamtzustand\\
%
\hline
Fachangestellte & Anspruch & Ausgabe &
 Informationsmenge auf Ihren Fachbereich fokussiert\\
  & Interesse & Eingabe &
  \\
%
\hline
\parbox[t]{3cm}{Fach- und\\Verwaltungsangestellte} & Anrecht & Dokumente &
 Reduktion der zu verarbeitenden Menge bei gleichbleibender Produktivität\\
 & Interesse & Verarbeitungsschritte &
 Reduktion\\
 & Anrecht & Abstimmung mit Kollegen &
 Reduktion des Aufwands fuer Abstimmung \\
%
\hline
Entscheider  &	Anspruch		&	Prozesszustand	&
Entscheidungsrelevante Informationen zu erhalten\\
%
 			& Anspruch & Prozesssteuerung &
flexibel, zeitnah und wirkungsvoll in Abläufe eingreifen\\
%
 & Interesse & \parbox[t]{4cm}{Schnittstellen zw\\Angestellten und System} & 
produktive Nutzung\\
 & Interesse & Schnittstellen zw Angestellten &
 effizienter, fehlerarmer Informationsfluss\\
%
\hline

\end{tabular}

\end{center}






\end{document}

