\chapter{Risiken}

\begin{comment}
Risiken können Ereignisse sein, die den Projektverlauf auf eine bestimmte Art und Weise gefährden könnten. Relevant sind dabei projektspezifische Risiken. Es ist wichtig sie vorab zu identifizieren und entsprechende Maßnahmen zu planen, um dieses Risiko zu minimieren bzw. den Umgang zu beschreiben, falls dieses Risiko auftritt. Technische Risiken sollten mit Hilfe eines Proof of Concept minimiert werden. Dementsprechend ist es wichtig zu beschreiben wie die Risiken mit einem PoC adressiert werden. 
\end{comment}



%
\section{Architektur}
\begin{enumerate}
\item Paradigma das Lose Kopplung unterstützt 
\item Steuerung der Priorisierung der Geschäftsobjekte
-> imb esten Fall unabhängig von Requests durch Clients
\end{enumerate}

%
\section{Technisch}
\begin{enumerate}
\item Mobiler Client
\begin{enumerate}
\item kommunikation zu steuerungskomponente im Verwaltungsdienst
\item Implementierung von interaktionsparadigmen (?)
\end{enumerate}
\item 
\end{enumerate}





\begin{comment}
%
\section{Marktzugang}
\paragraph*{Direkt}
Ein Integration in einen bestehenden Prozess einer Firma könnte schwierig sein wenn Firmen dieses Projekt als zu fragmentiert ansehen und lieber Lösungen "aus einer Hand" haben wollen.
%
\paragraph*{Partner}
Beim Marktzugang über Partner wie DMS Hersteller und Systemhäuser könnte die \brand Kernfunktionalität der Attributierung nicht als Mehrwert schaffende Zusatzkomponente angesehen werden.
%
\section{Fachlich}
Firmeninterne oder gesetzliche Regeln zur Dokumentenverarbeitung könnten den Verwaltungsaufwand der einen Attrbutierungsprozess umgibt 
unvorhergesehen erhöhen.
\end{comment}

