\chapter{Proof of Concepts}

\begin{comment}
Die Proof of Concepts lassen sich evtl. aus den Risiken ableiten. Für die Spezifizierung der Proof of Concepts müssen jeweils Exit- und Failkriterien beschrieben werden. D.h. es werden konkrete Bedingungen spezifiziert, die besagen in welchem Fall ein Proof of Concept als "erfolgreich" oder als "nicht erfolgreich" gilt. Falls ein Proof of Concept gescheitert ist, muss man sich im Vorfeld Alternativen/Fallbacks überlegen, die anstelle der ursprünglich angedachten Vorgehensweise herangezogen werden könnten. Die Durchführung eines Proof of Concepts muss dokumentiert werden. 
\end{comment}


\section{Definition}


\subsection{Architektur}

\paragraph*{Zugriffstruktur}



\subsection{Regel-Engine}
Ein simpler Poc der folgenden Ansprüchen erfüllt
\begin{enumerate}

\end{enumerate}

%
\subsection{mobiler Client}
%
Bezug: \ref{p2_risiken_mobil}{Risiken 2.1}
Erfolgskriterien:
\begin{itemize}
\item ruft Info api des Verwaltungsdienstes auf
\item erhält Informationen zum Systemzustand\\
\item stellt diese Informationen dar
\item ruft Steuerungs api mit Priorisierungsparameter auf
\end{itemize}
\noindent
Alternative:\\
Steuerungsfunktion als Browserbasierten Client umsetzen

%
\subsection{technisch: Steuerung & Dienst}


\subsection{•}


\section{Durchführung}


