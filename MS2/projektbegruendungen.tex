\chapter{Projektbegründungen}

\begin{comment}
Die Projektbegründungen sind jene Begründungen, die Bezug nehmen auf jegliche Entscheidungen, die im Projekt getroffen werden. Es sind somit projektspezifische Begründungen. Darin sollten Alternativen abgewägt werden und Inhalte auf den Punkt gebracht werden, sodass "Totes Wissen" eliminiert wird. Ein roter Faden sollte ersichtlich sein. Als Referenz dienen jeweilige Artefakte, die in dem Projekt entwickelt worden sind und demnach begründet werden müssen. 
\end{comment}


\section{Implementierung}
Die Entscheidung der Implementierungsumgebung wird aus den strategischen Zielen \ref{sec:zielhierarchie-strategisch}
sowie Punkt 3 der Kursziele abgeleitet, der da lautet:
\begin{verbatim}
Für die Bewerbungen in Unternehmen oder an Hochschulen ist heute oft neben einer 
guten Abschlussnote auch das Vorstellen einer anspruchsvollen, gut ausgeführten Projektarbeit 
ein wesentliches Erfolgskriterium. Das Praktikum hat das Ziel, den Studierenden die Möglichkeit 
zu geben, eine solche Arbeit zu erstellen oder zumindest einen ersten signifikanten 
Zwischenschritt bei Erstellung einer solchen Projektarbeit zu erreichen.
\end{verbatim}
%Quelle: \small{https://www.medieninformatik.th-koeln.de/w/Entwicklungsprojekt_interaktive_Systeme, abgerufen am 13.10.2015}
\noindent

Daraus folgt die Erkenntnis das eine fachliche und technologische Annäherung des Projekts an den antizipierten beruflichen Kontext das Ausmaß der Zielerfüllung des Kurses erhöht

\paragraph{beruflicher technologischer Kontext}

Im beruflichen Kontext wird für Windows Desktop und Windows Server im Stack .NET, \verb+c#+, MSSql entwickelt.

\paragraph{Risiken}
Eine Entwicklung im og Kontext würde folgende Nachteile mit sich bringen:
\begin{enumerate}
\item fehlende Unterstützung bei Implementierung durch Kursbetreuer
\item fehlende Portierbarkeit der Komponenten
\item ...
\end{enumerate}
\noindent

%@todo ggf raus
\paragraph{Chancen}

\begin{enumerate}
\item höhere Bewegungssicherheit im beruflich relevanten technologischen Kontext
%@todo besser formulieren
\item Wettbewerbsvorteil durch Erwerb technologischer Kompetenzen 'abseits der Masse'
\end{enumerate}

\paragraph{Entscheidung}\\
Daraus folgt die Entscheidung im beschriebenen technologischen Kontext zu implementieren. Es bleibt jedoch der Vorbehalt 
bei Bedarf einzelne Systemkomponenten in einem anderen Kontext zu implementieren.
%
%
\section{Objektbereich}
Aus der \nameref{sec:domaene-strukturierungsgrad} sowie Punkt 1 und 2 der \nameref{sec:zielhierarchie-strategisch} folgt
die Entscheidung das im Rahmen dieses Projekts der Objektbereich auf den Dokumenttyp Rechnung eingegrenzt wird.


%
\section{Nutzermodelle}
Auf die Rückmeldung der Betreuer, siehe \href{https://github.com/thuascgn/EISWS1516Howe/blob/master/MS1/Sprechstunde_MS1_20151012.md}{vom 12.10.2015, Punkt Rückmeldung}, den MCI relevanten Anteil zu erhöhen wird folgendermaßen reagiert:

%
\paragraph*{Variante 1: Nutzermodelle spezialisieren}\\
Eine Spezialisierung der Nutzermodelle würde einen Konflikt mit den strategischen Zielen bedeuten \ref{sec:zielhierarchie-strategisch} die eine möglichst breite wirtschaftliche Anwendungsdomäne anvisieren.\\
Zudem werden auch in spezialisierteren Anwendungsgebieten allgemeingültige Objektbereiche verarbeitet, siehe \ref{sec:codia_hochschulen}{quelle: codia an hochschulen}.

%
\paragraph*{Variante 2: Nutzungsmodelle erhöhen}\\
Um den MCI Anteil über zusätzliche Funktionen(?)/Nutzungsmodelle zu erhöhen muss ein Modell ausserhalb derer in \href{}{Anwendungsdomaene Nutzung} betrachteten gefunden werden. Dies böte die Chance eine größere Bandbreite der Nutzerinteraktion und 
Nutzungskontexte zu bearbeiten, bärge jedoch die Risiken des zusätzlichen Entwicklungsaufwandes für einen andersartigen Client
sowie ggf. erhöhte Planungsunsicherheit aufgrund der Nutzung von weniger geübten Technologien.

%
\paragraph*{Entscheidung}\\
Die Entscheidung fällt zugunsten der Variante 2, die in \href{sec:systembeschreibung_steuerungsclient}{Systembeschreibung} umschrieben
und in die Artefakte aufgenommen wird.


\section{Vorgehensmodell}


