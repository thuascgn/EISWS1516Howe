\chapter{Projektbegründungen}

\begin{comment}
Die Projektbegründungen sind jene Begründungen, die Bezug nehmen auf jegliche Entscheidungen, die im Projekt getroffen werden. Es sind somit projektspezifische Begründungen. Darin sollten Alternativen abgewägt werden und Inhalte auf den Punkt gebracht werden, sodass "Totes Wissen" eliminiert wird. Ein roter Faden sollte ersichtlich sein. Als Referenz dienen jeweilige Artefakte, die in dem Projekt entwickelt worden sind und demnach begründet werden müssen. 
\end{comment}



\section{Implementierung}

Die Entscheidung der Implementierungsumgebung wird aus den strategischen Zielen \ref{sec:zielhierarchie-strategisch}
sowie Punkt 3 der Kursziele abgeleitet, der da lautet:
\begin{verbatim}
Für die Bewerbungen in Unternehmen oder an Hochschulen ist heute oft neben einer 
guten Abschlussnote auch das Vorstellen einer anspruchsvollen, gut ausgeführten Projektarbeit 
ein wesentliches Erfolgskriterium. Das Praktikum hat das Ziel, den Studierenden die Möglichkeit 
zu geben, eine solche Arbeit zu erstellen oder zumindest einen ersten signifikanten 
Zwischenschritt bei Erstellung einer solchen Projektarbeit zu erreichen.
\end{verbatim}
%Quelle: \small{https://www.medieninformatik.th-koeln.de/w/Entwicklungsprojekt_interaktive_Systeme, abgerufen am 13.10.2015}
\noindent
\begin{comment}
%%%wilde herleitung :)
Ausgangszustand: aktuelle Situation des Teilnehmenden\\
Zielzustand: Punkte 1, 2 und 3 der Kurszielsetzung 
Erfolg: Evaluation des Projektes anhand des Erfüllungsgrades der Kursziele\\
Punkt 3 der Kursziele lässt sich als Schnittstelle des Teilnehmers zu potentiellen Arbeitgebern
definieren an welche der Kursteilnehmer die Parameter seiner Fähigkeiten an den Arbeitgeber übergibt, 
diese auf Kompatibilität geprüft werden und bei hinreichender Übereinstimmung eine Zusammenarbeit zustande kommt.\\
\end{comment}

Daraus folgt die Erkenntnis das eine fachliche und technologische Annäherung des Projekts an den antizipierten beruflichen Kontext das Ausmaß der Zielerfüllung des Kurses erhöht
% und \item je genauer der berufliche Kontext antizipiert werden kann, desto besser kann das Projekt diesem angenähert werden.\end{enumerate}

\paragraph{beruflicher technologischer Kontext}

Im beruflichen Kontext wird für Windows Desktop und Windows Server im Stack .NET, c\n#, MSSql entwickelt.


\paragraph{Contra}
Eine Entwicklung im og Kontext würde folgende Nachteile mit sich bringen:

\begin{enumerate}
\item fehlende Unterstützung bei Implementierung durch Kursbetreuer
\item fehlende Portierbarkeit der Komponenten
\item ...
\end{enumerate}


%@todo ggf raus
\paragraph{Pro}

\begin{enumerate}
\item höhere Bewegungssicherheit im beruflich relevanten technologischen Kontext
%@todo besser formulieren
\item Wettbewerbsvorteil durch Erwerb technologischer Kompetenzen 'abseits der Masse'
\end{enumerate}

\paragraph{Entscheidung}\\
Daraus folgt die Entscheidung im beschriebenen technologischen Kontext zu implementieren. Es bleibt jedoch der Vorbehalt 
bei Bedarf einzelne Systemkomponenten in einem anderen Kontext zu implementieren.
%
%
\section{Objektbereich}

Aus der \nameref{sec:domaene-strukturierungsgrad} und der Punkt 1 und 2 der \nameref{sec:zielhierarchie-strategisch} folgt
die Entscheidung das im Rahmnen dieses Projekts der Objektbereich auf den Dokumenttyp Rechnung eingegrenzt wird.


\section{Nutzermodelle}


