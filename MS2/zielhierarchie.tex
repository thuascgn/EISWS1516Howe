\chapter{Zielhierarchie}
\label{cha:zielhierarchie}

\begin{comment}
Eine Zielhierarchie lässt sich in drei Ebenen strukturieren.
+ strategische Ziele sind Ziele, die auf langfristige Sicht erreicht werden sollen. 
	Strategische Ziele beantworten die Frage "Was soll erreicht werden?"
+ taktische Ziele sind Ziele, die auf mittelfristige Sicht erreicht werden sollen.
	Taktische Ziele beantworten die Frage "Wie soll es erreicht werden?"
+ operative Ziele sind Ziele, die auf kurzfristige Sicht erreicht werden sollen.
	Operative Ziele beantworten die Frage "durch welche Aktivitäten soll es erreicht werden?" 
Die Zielpriorisierung sollte sich durch die verwendeten Verben (muss,soll,kann) ausdrücken. 
\end{comment}


\section{Strategisch}
\label{sec:zielhierarchie-strategisch}

\begin{enumerate}
\item Anwendungskontext:\\
Es soll ein Anwendungskontext mit möglichst hoher wirtschaftlicher Relevanz gefunden werden
\item Technologisch:\\
Es sollen möglichst viele im beruflichen Kontext relevanten Erfahrungen... , siehe \nameref{}
\item Objektbereich:\\
Es muss ein Komplexitätsgrad erreicht werden der fachlich relevant und technologisch, im Rahmen des Projekts, beherrschbar ist
\item Nutzung:\\
Anwender sollen vom Ballast repitiver Aufgaben befreit werden
\item 
\end{enumerate}


%
\section{Taktisch}
\label{sec:zielhierarchie-taktisch}

\begin{enumerate}
\item Objektbereich:\\
Es muss eine Ananlyse und Bewertung der in der Anwendungsdomäne genutzen Objekte, dh. dokumentenklassen, durchgeführt werden
\item Nutzung:\\
Es muss Automatisierungspotential identifiziert werden
\item ...
\item ...
\end{enumerate}


%
\section{Operational}
\label{sec:zielhierarchie-operational}

\begin{enumerate}
%
\item Nutzung:\\
\begin{enumerate}
\item deskriptive Aufgabenanalyse
\item Automatisierungspotential identifizieren
\item präskriptive Aufgabenanalyse, inkl P2
\end{enumerate}
%
\item 
\end{enumerate}



