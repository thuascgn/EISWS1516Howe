\chapter{Zielhierarchie}
\label{cha:zielhierarchie}

\begin{comment}
Eine Zielhierarchie lässt sich in drei Ebenen strukturieren.
+ strategische Ziele sind Ziele, die auf langfristige Sicht erreicht werden sollen. 
	Strategische Ziele beantworten die Frage "Was soll erreicht werden?"
+ taktische Ziele sind Ziele, die auf mittelfristige Sicht erreicht werden sollen.
	Taktische Ziele beantworten die Frage "Wie soll es erreicht werden?"
+ operative Ziele sind Ziele, die auf kurzfristige Sicht erreicht werden sollen.
	Operative Ziele beantworten die Frage "durch welche Aktivitäten soll es erreicht werden?" 
Die Zielpriorisierung sollte sich durch die verwendeten Verben (muss,soll,kann) ausdrücken. 
\end{comment}


\section{Strategisch}
\label{sec:zielhierarchie-strategisch}

\begin{enumerate}
\item Objektbereich:\\
Es muss ein Komplexitätsgrad erreicht werden der fachlich relevant und technologisch, im Rahmen des Projekts, beherrschbar ist
\item Anwendungsdomäne:\\
Es soll ein Anwendungskontext mit möglichst hoher wirtschaftlicher Relevanz gefunden werden
\item Technologisch:\\
Es sollen möglichst viele im beruflichen Kontext relevanten Erfahrungen... , siehe \nameref{}
\item Nutzung:\\
Anwender sollen vom Ballast repetiver Aufgaben befreit werden
\end{enumerate}


%
\section{Taktisch}
\label{sec:zielhierarchie-taktisch}

\begin{enumerate}
\item Anwendungskontext \& Objektbereich:\\
Es muss eine qualifizierte Entscheidungsgrundlage geschaffen werden
\item Technologisch:\\
Implementierungrelevante Entscheidungen sollen gegen den beruflichen Kontext bewertet werden
\item Nutzung:\\
Es soll Automatisierungspotential identifiziert werden
\item ...
\end{enumerate}


%
\section{Operational}
\label{sec:zielhierarchie-operational}

\begin{enumerate}
%
\item Anwendungskontext \& Objektbereich:\\
Es muss eine Analyse und Bewertung der in den Anwendungsdomänen genutzen Objekte, dh. Dokumentenklassen, durchgeführt werden
\item Technologisch:\\
Begründen wenn vom definierten Standard \href{sec:entscheidungen} abgewichen wird
\item Nutzung:\\
\begin{enumerate}
\item deskriptive Aufgabenanalyse
\item Automatisierungspotential identifizieren
\item präskriptive Aufgabenanalyse, inkl P2
\end{enumerate}
\item ...
\end{enumerate}



