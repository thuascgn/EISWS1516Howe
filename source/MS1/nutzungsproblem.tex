\chapter{Nutzungsproblem}

In Unternehmen werden digitale und analoge Eingangsdokumente, wie Rechnungen oder Lieferscheine, 
in einem Erfassungsprozess in eine vorbereitenden Struktur gebracht. Danach liegen die Dokumente als Dokumentendatei, Bilddatei sowie beschreibende xml-Datei vor und bilden die Grundlage für das zu verarbeitende Geschäftsobjekt. Abhängig von Dokumententyp und Ausprägung werden dem Geschäftsobjekt in weiteren Prozessen, zb. Buchung einer Rechnung, durch eine bürokaufmännische Fachkraft weitere Attribute zugewiesen.\\
%nur in den Werten und nicht in den Attributen die zugewiesen werden.
Ein Großteil dieser Zuweisungen ist sehr repetitiv und variiert nur geringfügig. Daher wird dieser Prozess über eine Regel-Engine automatisiert, bei der die bürokaufmännische Fachkraft der Regel-Engine mitteilt welche Werte für eine Dokumentenausprägung zugewiesen werden sollen.\\
Es kann zudem vorkommen das diese Person nicht ausreichend fachbezogenes Wissen besitzt um eine Zuweisung durchzuführen. Dafür wendet diese sich an einen zuständigen fachkundigen Kollegen. Dies geschieht in der Regel über seperate Kanäle (Telefon, EMail, ...), die ausserhalb des eigentlichen Prozesses verlaufen, was eine Reihe von Nachteilen mit sich bringt. Um diesen Nachteilen entgegenzuwirken wird ein Fachclient zur Verfügung gestellt an den ein Zuweisungsvorgang abgegeben werden kann.