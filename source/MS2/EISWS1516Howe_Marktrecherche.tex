\documentclass[11pt,oneside,a4paper,notitlepage]{article}

\usepackage[utf8]{inputenc}
\usepackage[ngerman]{babel}
\usepackage[margin=1.5cm]{geometry}

%kommentare, zitate, quellcode
\usepackage{verbatim}
%\fontfamily{sfdefault}
\renewcommand{\familydefault}{\sfdefault}
%
\usepackage{graphicx}
%fuer tabellen
\usepackage{tabularx}
\usepackage{tabulary}
%
%formatierung listen
\let\oldenumerate\enumerate
\renewcommand{\enumerate}{
  \oldenumerate
  \setlength{\itemsep}{1pt}
  \setlength{\parskip}{0pt}
  \setlength{\parsep}{0pt}
}

%
%referenzen und links
\usepackage{hyperref}
\hypersetup{
colorlinks=false,
hidelinks=false
}
%
% 
\renewcommand{\arraystretch}{1.5}
\newcommand{\brand}{Projektsystem }
%

\begin{document}
%
\section{Marktrecherche}

\begin{comment}
Die Marktrecherche, auch related works genannt, beinhaltet die Recherche nach konkurrierenden Systemen, die teilweise oder vollständig die Funktionalitäten aufweisen wie sie für das zu entwickelte System geplant sind. Dabei sollte man Vor- und Nachteile gegenüber dem zu entwickelten System herausstellen. 
\end{comment}

Der Markt wird im allgemeinen von Entwicklern einzelner Komponenten und Systemhäusern bestimmt die Fremdkomponenten ggf mit Eigenentwicklungen kombinieren und so individuelle Lösungspakete schnüren welche sich immer an den in beschrieben etablierten \ref{sec:anwendungsdomaene_prozess}{Prozess} der Anwendungsdomäne orientieren.\\
Die Komponenten lassen sich grob in die folgenden Bereiche enteilen:
\begin{itemize}
\item Capture: Scannen, OCR, klassifizieren
\item Workflow: Attributierung
\item betriebliche Anwendungssysteme: Buchhaltung, Archivierung
\end{itemize}
\noindent
Zudem sind OEM Versionen \href{}{glossar} durchaus gängige Praxis wodurch eine Einsicht erschwert wird.\\
Um dennoch einen Eindruck in die Marktsituation zu gewinnen hilft eine Betrachtung der og Bereiche in der funktional übergeordneten Ebene der DMS-Systeme, ... 
% DMS-Sys.. Und??? präziser beschreiben
%
und deren Eigenschaften. Es werden Ansätze zweier Anbieter exemplarisch beschrieben und eine Einordnung zwischen diesen Lösungen und \brand mittels einer Featurematrix ermöglicht.


\subsection{Codia DMS }
%
% http://www.codia.de/codia/files/dms_in_hochschulenv1.pdf
%
% http://www.codia.de/codia/d.3ecm--die-basis.php
Codia DMS bietet auf Basis des d3.ecm von d.velop spezialisierte Lösungen im eGovernment Umfeld für öffentliche Verwaltung und Hochschulen mit den Themen Scannen & Klassifizierung, Rechungs- und Eingangspostverarbeitung, eAkte und Archivierung.


\subsection{InPunkto}
% http://www.inpuncto.com/de/sap-ecm-loesungen/prozessoptimierung-sap-workflow/elektronische-rechnungsbearbeitung/automatisierter-rechnungseingang/
InPunkto spezialisert auf Dokumenten Dienstleistungen im SAP Umfeld mit den Themen automatische Erfassung \& Verarbeitung, Workflow, 
eAkte und Archivierung.


\subsection{Übersicht}
%http://www.codia.de/codia/workflow.php
%http://www.codia.de/codia/eingangsrechnungsverarbeitung.php
%

%\begin{comment}
\begin{center}
\begin{tabulary}{\textwidth}{LLLL}
Thema & Codia & InPunkto & \brand \\
Automatisierte Klassifizierung & J & J & N \\
Attributierung & & & \\
semantisch & J & kA & N\\
fachlich & J & J & J \\
Automatisierte Attributierung & N & N & J \\
Steuerung & lastabhängige Aufgabenverteilung\newline über Workflowsystem & kA & Priorisierung\\
(Rechnungs) Workflow & d3ecm & SAP Workflow & freie Wahl \\
Export & d3.ecm & SAP & Rohexport als xml\\
\end{tabulary}
\end{center}
%\end{comment}



\subsection{Quellen}


\label{q:1} ...

%
\end{document}

