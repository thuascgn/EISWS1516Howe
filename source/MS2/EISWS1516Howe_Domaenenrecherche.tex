\documentclass[11pt,oneside,a4paper,notitlepage]{article}

\usepackage[utf8]{inputenc}
\usepackage[ngerman]{babel}
\usepackage[margin=1.5cm]{geometry}

%kommentare, zitate, quellcode
\usepackage{verbatim}
%\fontfamily{sfdefault}
\renewcommand{\familydefault}{\sfdefault}
%
\usepackage{graphicx}
%fuer tabellen
\usepackage{tabularx}
\usepackage{tabulary}
%
%formatierung listen
\let\oldenumerate\enumerate
\renewcommand{\enumerate}{
  \oldenumerate
  \setlength{\itemsep}{1pt}
  \setlength{\parskip}{0pt}
  \setlength{\parsep}{0pt}
}

%
%referenzen und links
\usepackage{hyperref}
\hypersetup{
colorlinks=true,
linkcolor=cyan,
urlcolor=cyan,
hidelinks=false
}
%
% 
\renewcommand{\arraystretch}{1.5}
%

\begin{document}
%

\section{Domänenrecherche}
\label{cha:domaene}

\begin{comment}
Die Domänenrecherche zeigt auf in welcher Domäne das System zukünftig eingesetzt werden soll. Dabei soll recherchiert werden welche wichtigen Konzepte der Anwendungsdomäne eine Rolle bei der Gestaltung des Systems spielen. Es werden zudem sämtliche Informationen, Vorgänge, Metaphern und Paradigmen aus der Domäne recherchiert, in dem sich das zu entwickelnde System befindet. Die Domänenrecherche bildet die Basis für die Entwicklung eines Nutzungsproblems und der nachfolgenden Konzeption des gesamten Systems. 
\end{comment}

Im folgenden werden die wichtigsten Themen der Anwendungsdomäne beschrieben und einer ersten Bewertung
hinsichtlich projektrelevanter Eigenschaften unterzogen.



\subsection{Allgemeiner Verarbeitungprozess}
\label{sec:domaene-prozess}

Beispielhaft die wichtigsten funktionalen Komponenten des Dokumentenverabreitungsprozesses beschrieben \ref{q:riggert}{[1, S.13]}, wichtig bleibt anzumerken das in konkreten Implementierungen die Funktionalitäten verschwimmen können und keine klare Trennung wie hier beschrieben realisiert ist. 
%
% wichtig für Architektur > Marktrecherche(?)
%

\begin{enumerate}
\item Extraktion:\\
Manuelles oder mittels OCR automatisiertes auslesen von Informationen aus einer Dokumentendatei oder einer zugehörigen Bitmapdatei.
\begin{enumerate}
\item optional Nacherfassung:\\
Kontrolle der OCR Ergebnisse und ggf Korrektur bei unzureichender Extraktionsqualität
\end{enumerate}
\item Klassifizierung:\\
Beschreibt die Einordnung in eine klar abgegrenzte Menge von Dokumenttypen wie zb. Formulare, Rechnungen, Lieferscheine, 
Bewerbungen.
\item Attributierung:\\
Beschreibt den Prozess der Zuordnung von organisations- oder fachspezifischen Attributen zu einem Dokument zur weiteren
Verarbeitung innerhalb der Organisation. Die können zum Beispiel Buchungskonten, Kostenstellen, Projektnummern oder Anprechpartner   sein.
\item Export \& weitere Verarbeitung:\\
Übergabe der klassifizierten und attributierten Dokumente an einem betriebliches Anwendungssystem zur Buchung, Kontierung oder Archivierung.
\end{enumerate}
\noindent

%
\subsection{Begrifflichkeiten}

\subsubsection{Attributierung}
In der Praxis wird der Begriff der Attributierung für zweierlei Aspekte verwendet:\\
%, häufig unscharf definiert 
\begin{enumerate}
\item semantische(?) Attributierung:\\
Extrahierung und Zuordnung von Metainformationen zu einem Dokument welche sich auf das Dokument, bzw die Dokumentendatei als Repräsentation des Dokuments, selbst beziehen. Dies können zb Schlagworte, Speicherort ... sein.
\item fachliche Attributierung:\\
Die fachliche Attributierung ist ein Teil des Arbeitsprozesses bei welchem dem Dokument Attribute zugeordnet werden die 
Diese Attribute können zb kaufmännischer, steuerlicher oder juritischer Natur sein.
\end{enumerate}
\noindent
Für dieses Projekt wird der Begriff im Sinne der fachlichen Attributierung verwendet.

\subsubsection{Strukturierungsgrad}
\label{sec:domaene-struturierungsgrad}

Der Strukturierungsgrad eines Dokuments wird beschrieben durch das Ausmaß an Sicherheit mit der ein Wert eines Dokumentenattributs
an einer Position im Dokument auftritt und welcher Wertebereich in diesem abgebildet wird.
	
\begin{enumerate}
\item unstrukturiert: gar keine bis geringe Positionssicherheit mit überwiegend undefiniertem Wertebereich, zb. Bewerbungsschreiben
\item semi-strukturiert: gute Positionssicherheit mit überwiegend definiertem Wertebereich, zb. Rechnungen, Lieferscheine
\item strukturiert: absolute Positionssicherheit mit klar definiertem Wertebereich, zb. genormte betriebliche oder behördliche Formulare
\end{enumerate}
\noindent
%Je höher der Strukturierungsgrad eines Dokumenttyps desto spezialisiertier besser eignet sich dieser für eine automatisierte Verarbeitung.
Mit dem Strukturierungsgrad steigt die semantische Spezialisierung sowie das Automatisierungspotential für diesen Dokumenttyp.

%
\subsubsection{Geschäftsobjekt}
Im Kontext der Anwendungsdomäne wird das zentrale verarbeitete Geschäftsobjekt aus einer Menge von Dokumentendatei und zugeordneten 
beschreibenden Daten gebildet. 
%@todo repräsentation verschieben?
Das Dokument liegt dabei in der Regel als Repräsentation im Dokumentformat .pdf, als .tiff oder .jpg Bitmapdatei und die beschreibenden
Daten als .xml Datei vor.
%

%
\subsection{Anwendungsfelder}
\label{sec:domaene-anwendungsfelder}

\begin{table}[ht]
\caption{Automatisierte Dokumentenverarbeitung}\\
  \begin{tabulary}{\textwidth}{LLLL}
Domäne 		& Primäre Dokumenttypen 	& Strukturierungsgrad 	& wirtschaftliche Relevanz  \\ 
\hline \\
Buchhaltung	& Rechnungen				& 2 						& ++\\ 
\hline
Verwaltung	& Formulare 				& 3 						& + \\
 \hline
Personal 	& Bewerbungen 			& 1-2 					& o \\ 
\hline
Kanzleien 	& Schriftverkehr			& 1						& + \\ 
\hline
Logistik 	& Lieferscheine 			& 2 						& + \\
%\hline Produktion	&	...					& 2						& + \\
\hline
Privat 		& Rechnungen\newline 
			Versicherungen\newline
			Formulare				& 2\newline
									  1\newline
									  3 						& -\newline
									  							Services für Privatanwender\newline nicht etabliert 
  \end{tabulary} 
\end{table}
\noindent

\newpage
\section{Nutzungsmerkmale}

Eine erste kurze Betrachtung der antizipierten Nutzungsmerkmale welche Entscheidungen zu Anwendungsdomäne (ref), Objektbereich (ref)
und ... unterstützen soll.

\begin{table}[ht]
\caption{Nutzungsmerkmale}\\
\begin{tabulary}{\textwidth}{LLL}
Domäne		& Arbeitsumgebung & Arbeitsgerät(?) & Organisationsrolle\\
\hline \\
Buchhaltung	& Büro & Desktop PC & Bürokfm Fachkraft \\
\hline
Verwaltung	& Büro & Desktop PC & Bürokfm oder Verwaltungs- Fachkraft \\
\hline
Personal 	& Büro & Desktop PC & Personaldienstleistungskfm Fachkraft\\ 
\hline
Kanzleien 	& Büro & Desktop PC & Bürokfm Fachkraft\\ 
\hline
Logistik 	& Büro\newline ggf Mobil, in größeren Betrieben & Desktop PC\newline ggf Tablet & Bürokfm Fachkraft, Lagerist\\
%\hline Produktion	& Werkhalle	& \\
\hline
Privat 		& Zuhause\newline Mobil & Desktop PC\newline Tablet, Smartphone & Privat, Organisieren von Post(?) \\ 
\end{tabulary}
\end{table}
\noindent
%

%
\subsection{Speziellere Anwendungsdomänen}

Wie unter anderen von Codia \href{q:codia-hs}{[1]} beschrieben ergeben sich die Bedürfnisse in spezielleren Anwendungsdomänen wie zb Hochschulen nur teilweise durch spezielle Dokumenttypen wie eine Studentenakte. Auch hier spielen allgmeinere Dokumenttypen wie Rechungen eine große Rolle.


\newpage
%
\subsection{Quellen}

\label{q:riggert} 1. Wolfgang Riggert, Klassifikation \& Extraktion, FH Flensburg abgerufen am 15.10.2015, url: \href{http://www.wi.fh-flensburg.de/fileadmin/dozenten/Riggert/bildmaterial/Dokumentenmanagement/2-Capture-Klassifikation___Extraktion.pdf}{Klassifikation \& Extraktion}
\\ \\
\noindent
\label{q:codia-hs} 2. Codia DMS in Hochschulen, abgerufen am 15.10.2015, url: \href{http://www.codia.de/codia/files/dms_in_hochschulenv1.pdf}{DMS in Hochschulen}



%
\end{document}

