\documentclass[11pt,oneside,a4paper,notitlepage]{article}

\usepackage[utf8]{inputenc}
\usepackage[ngerman]{babel}
\usepackage[margin=1.5cm]{geometry}

%kommentare, zitate, quellcode
\usepackage{verbatim}
%\fontfamily{sfdefault}
\renewcommand{\familydefault}{\sfdefault}
%
\usepackage{graphicx}
%fuer tabellen
\usepackage{tabularx}
\usepackage{tabulary}
%
%formatierung listen
\let\oldenumerate\enumerate
\renewcommand{\enumerate}{
  \oldenumerate
  \setlength{\itemsep}{1pt}
  \setlength{\parskip}{0pt}
  \setlength{\parsep}{0pt}
}

%
%referenzen und links
\usepackage{hyperref}
\hypersetup{
colorlinks=true,
linkcolor=cyan,
urlcolor=cyan,
hidelinks=false
}
%
% 
\renewcommand{\arraystretch}{1.25}
%

\begin{document}
%

\section{User Profiles}


\section{Importeur}

\paragraph*{Aufgaben: }
\begin{itemize}
\item öffnen von Post
\item einscannen von relevanten Dokumenten (filtern von Werbung oä.)
\end{itemize}
%
%\subparagraph*{primäre: }
%
\paragraph*{Qualifikation: } keine
%
\paragraph*{Berufserfahrung: } keine bis wenig
%
\paragraph*{Arbeitssoftware: } Scan Software: idR. von Hersteller der Scan Hardware, ggf. Kontrolle der Scanqualität über
Bitmapanzeige Software oder Scan Software
%
\paragraph*{Arbeitshardware: } Scan Station: Scanner, ggf. PC und Bildschirm

%
%
\section{Kontrolleur}
\paragraph*{Aufgaben: }
\begin{itemize}
\item kontrollieren des OCR-Erkennungsergebnisses eines Dokuments
\item ggf korrigieren oder ergänzen des Ergebnisses
\item ggf klassifizieren des Dokuments
\end{itemize}
%
%\subparagraph*{primäre: }
%
\paragraph*{Qualifikation: } keine
%
\paragraph*{Berufserfahrung: } keine bis wenig
%
\paragraph*{Arbeitssoftware: } Erfassungssoftware; Dokumentanzeige, Anzeige des OCR Ergebebnisses, Eingabemöglichkeit zur
Korrektur des OCR-Erkennungsergebnisses
%
\paragraph*{Arbeitshardware: } Büroarbeitsplatz mit PC


%
%
\section{Buchhaltungsangestellte}
%
\paragraph*{Aufgaben: }
\begin{itemize}
\item ggf Korrektur der Erfassung
\item Übersicht und Verwaltung des Dokumenteneingangs
\item Erzeugen eines Geschäftsvorgangs auf Basis von Dokumenten
\item Zuweisung von buchhalterischen oder verwaltungsspezifischen Attributen zu einem Dokument
\item übergabe des Dokuments an einen Fachangestellten
\item Absprachen mit fachangestellten Kollegen
\item Berichten und Absprachen mit Vorgesetzten
\end{itemize}
\noindent

\paragraph*{Qualifikation: } Verwaltungsbezogene oder Bürogkfm. Ausbildung, seltener auch Studium
%
\paragraph*{Berufserfahrung: } wenige bis viele Jahre
%
\paragraph*{Arbeitssoftware: } Erfassungssoftware; Dokumentanzeige, Eingabemöglichkeit zur Korrektur Erkennungsergebnisses, 
betriebliches Anwendungssystem für zb Buchhaltung oder Geschäftsvorgänge/Workflow

%
\paragraph*{Arbeitshardware: } Büroarbeitsplatz mit PC

%
%
\section{Fachangestellte}
%
\paragraph*{Alter: }
\
%
\paragraph*{Qualifikation: } fachbezogenes Studium oder Ausbildung
%
\paragraph*{Berufserfahrung: } wenige bis viele Jahre
%
\paragraph*{Arbeitssoftware: } betriebliches Anwendungssystem, Office-Paket, ERP-Systeme (ggf teilweise webbasiert), ggf fachbezogene Spezialsoftware
%
\paragraph*{Arbeitshardware: } Büroarbeitsplatz mit PC
%
\paragraph*{Aufgaben: }
\begin{itemize}
\item bearbeiten Ihre fachlichen Kernbereichs
\item Arbeit in fachbezogenen oder Abteilungsübergreifenden Projekten
\item Zuweisung von fach- oder projektspezifischen Attributen zu einem Dokument
\item ggf Übergabe des Dokuments an einen weiteren Fachangestellten
\item Absprachen mit fachangestellten Kollegen
\item Berichten und Absprachen mit Vorgesetzten
\end{itemize}
\noindent





%
%
\section{Verantwortlicher Buchhaltung}

\paragraph*{Aufgaben: }
\begin{itemize}
\item Überblicken des Verwaltungsprozesses, zb über Besprechungen der Verwaltungsabteilungen oder Status-Berichte von Angestellten, evtl Einsicht in Prozesstati über betriebliches Anwendungssystem
\item Evaluieren der Berichte von Verwaltungsangestellten, ggf. korrigieren von Ungereimtheitem in Prozessabläufen
\item genehmigen von Geschäftsvorgängen
\item Lösen von Konflikten im Verwaltungsprozess
\item Lösen von Konflikten zwischen Verwaltung und Fachabteilungen, Absprachen mit Verantwortlichen von Fachabteilungen
\item Besprechungen mit Verantwortlichen von Fachabteilungen und Geschäftsführung. Ab mittlerer Unternehmensgröße auch oft an anderen Standorten
\item 
\end{itemize}
\noindent
%
\paragraph*{Qualifikation: } Studium wie zb. Controlling, auch Verwaltungsbezogene oder Bürokfm. Ausbildung
%
\paragraph*{Berufserfahrung: } idR mind 5 Jahre
%
\paragraph*{Arbeitssoftware: } betriebliches Anwendungssystem, Office-Paket, ERP-Systeme (ggf teilweise webbasiert)
%
\paragraph*{Arbeitshardware: } Büroarbeitsplatz mit PC, mobiles Endgerät

%
%
\section{Verantwortlicher Fachabteilung}

\paragraph*{übliche Fachabteilungen: } Personal, IT, Marketing, Sales/Vertrieb, Recht, Finanzen, Forschung und Entwicklung, Produktion, Logistik

\paragraph*{Aufgaben: }
\begin{itemize}
\item Überblicken der Prozesse und Projekte der Fachabteilung, zb über Besprechungen der Fachabteilung oder Status-Berichte von Angestellten, evtl. Einsicht in Prozesstati über betriebliches Anwendungssystem
\item Evaluieren der Berichte von Fachangestellten, ggf. korrigieren von Ungereimtheitem in Prozessabläufen
\item genehmigen von Geschäftsvorgängen
\item Lösen von Konflikten zwischen Fachabteilungen, durch Absprachen mit Verantwortlichen von andren Fachabteilungen
\item Besprechungen mit Verantwortlichen von Fachabteilungen und Geschäftsführung. Ab mittlerer Unternehmensgröße auch oft an anderen Standorten
\item Besprechungen mit externen Kontakten wie Kunden, Geschäftspartner
\end{itemize}
\noindent

\paragraph*{Qualifikation: } fachbezogenes Studium oder Ausbildung

%
\paragraph*{Berufserfahrung: } idR mind 5 Jahre

%
\paragraph*{Arbeitssoftware: } betriebliches Anwendungssystem, Office-Paket, ERP-Systeme (ggf teilweise webbasiert), ggf fachbezogene Spezialsoftware

%
\paragraph*{Arbeitshardware: } Büroarbeitsplatz mit PC, mobiles Endgerät


%

\section{•}
\end{document}

