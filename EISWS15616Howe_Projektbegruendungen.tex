\documentclass[11pt,oneside,a4paper,notitlepage]{book}

\usepackage[utf8]{inputenc}
\usepackage[ngerman]{babel}
\usepackage[margin=1.5cm]{geometry}

%kommentare, zitate, quellcode
\usepackage{verbatim}
%\fontfamily{sfdefault}
\renewcommand{\familydefault}{\sfdefault}
%
\usepackage{graphicx}
%fuer tabellen
\usepackage{tabularx}
\usepackage{tabulary}
%
%formatierung listen
\let\oldenumerate\enumerate
\renewcommand{\enumerate}{
  \oldenumerate
  \setlength{\itemsep}{1pt}
  \setlength{\parskip}{0pt}
  \setlength{\parsep}{0pt}
}

%
%referenzen und links
\usepackage{hyperref}
\hypersetup{
colorlinks=true,
hidelinks=false
}
%

%
\newcommand{\brand}{TaARs }
%
%\newcommand{\anm}[1]{\\ \begin{comment} #1 \end{comment}\\ }
%
% - - - - - - - - - - - - - - - - - - 
%
\begin{document}


\begin{center}
\Large{EIS WS1516 - Meilenstein 2}\\[3mm]
%\normalsize{Verteiltes Training einer automatisierten Dokumentenattributierung}\\[3mm]
\normalsize{Tim Howe}
\normalsize{Projektbegründungen}
\end{center}
%\brand - Verteilte Group- Middleware zum Training einer automatisierten Attributierung von Rechnungen


\tableofcontents


\newpage

\begin{comment}
Die Projektbegründungen sind jene Begründungen, die Bezug nehmen auf jegliche Entscheidungen, die im Projekt getroffen werden. Es sind somit projektspezifische Begründungen. Darin sollten Alternativen abgewägt werden und Inhalte auf den Punkt gebracht werden, sodass "Totes Wissen" eliminiert wird. Ein roter Faden sollte ersichtlich sein. Als Referenz dienen jeweilige Artefakte, die in dem Projekt entwickelt worden sind und demnach begründet werden müssen. 
\end{comment}


\newpage
\chapter{Methodischer Rahmen}

\begin{comment}
Der methodische Rahmen bezieht sich auf die Wahl des MCI Vorgehensmodells und sollte vorab definiert werden. Im methodischen Rahmen werden angemessene Methoden für jegliche Aktivitäten und Unteraktivitäten festgelegt, welche projektspezifisch (d.h. aus dem Problemraum heraus) definiert und begründet sind. Mögliche Iterationen sollten berücksichtigt werden.
\end{comment}
\newpage
\chapter{Zielhierarchie}
\label{cha:zielhierarchie}

\begin{comment}
Eine Zielhierarchie lässt sich in drei Ebenen strukturieren.
+ strategische Ziele sind Ziele, die auf langfristige Sicht erreicht werden sollen. 
	Strategische Ziele beantworten die Frage "Was soll erreicht werden?"
+ taktische Ziele sind Ziele, die auf mittelfristige Sicht erreicht werden sollen.
	Taktische Ziele beantworten die Frage "Wie soll es erreicht werden?"
+ operative Ziele sind Ziele, die auf kurzfristige Sicht erreicht werden sollen.
	Operative Ziele beantworten die Frage "durch welche Aktivitäten soll es erreicht werden?" 
Die Zielpriorisierung sollte sich durch die verwendeten Verben (muss,soll,kann) ausdrücken. 
\end{comment}


\section{Strategisch}
\label{sec:zielhierarchie-strategisch}

\begin{enumerate}
\item Objektbereich:\\
Es muss ein Komplexitätsgrad erreicht werden der fachlich relevant und technologisch, im Rahmen des Projekts, beherrschbar ist
\item Anwendungsdomäne:\\
Es soll ein Anwendungskontext mit möglichst hoher wirtschaftlicher Relevanz gefunden werden
\item Technologisch:\\
Es sollen möglichst viele im beruflichen Kontext relevanten Erfahrungen... , siehe \nameref{}
\item Nutzung:\\
Anwender sollen vom Ballast repetiver Aufgaben befreit werden
\end{enumerate}


%
\section{Taktisch}
\label{sec:zielhierarchie-taktisch}

\begin{enumerate}
\item Anwendungskontext \& Objektbereich:\\
Es muss eine qualifizierte Entscheidungsgrundlage geschaffen werden
\item Technologisch:\\
Implementierungrelevante Entscheidungen sollen gegen den beruflichen Kontext bewertet werden
\item Nutzung:\\
Es soll Automatisierungspotential identifiziert werden
\item ...
\end{enumerate}


%
\section{Operational}
\label{sec:zielhierarchie-operational}

\begin{enumerate}
%
\item Anwendungskontext \& Objektbereich:\\
Es muss eine Analyse und Bewertung der in den Anwendungsdomänen genutzen Objekte, dh. Dokumentenklassen, durchgeführt werden
\item Technologisch:\\
Begründen wenn vom definierten Standard \href{sec:entscheidungen} abgewichen wird
\item Nutzung:\\
\begin{enumerate}
\item deskriptive Aufgabenanalyse
\item Automatisierungspotential identifizieren
\item präskriptive Aufgabenanalyse, inkl P2
\end{enumerate}
\item ...
\end{enumerate}




\newpage
\input{./doc/anforderungsermittlung.tex}
\newpage
\input{./doc/loesungen.tex}
\newpage
\input{./doc/weiteres.tex}


%
%
%
\end{document}